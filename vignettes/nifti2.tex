\documentclass[
]{article}

\usepackage[utf8]{inputenc}

\providecommand{\tightlist}{%
  \setlength{\itemsep}{0pt}\setlength{\parskip}{0pt}}

\author{
Brandon Whitcher\\Pfizer Worldwide R\&D \And Volker J. Schmid\\Ludwig-Maximilians Universit"at M"unchen \And Andrew Thornton\\Cardiff University
}
\title{Working with the NIfTI Data Standard in \proglang{R}}

\Plainauthor{Brandon Whitcher, Volker J. Schmid, Andrew Thornton}
\Plaintitle{Working with the NIfTI Data Standard in R}
\Shorttitle{NIfTI and R}

\Abstract{
The package \pkg{oro.nifti} facilitates the interaction with and
manipulation of medical imaging data that conform to the ANALYZE, NIfTI
and AFNI formats. The \proglang{S}4 class framework is used to develop
basic ANALYZE and NIfTI classes, where NIfTI extensions may be used to
extend the fixed-byte NIfTI header. One example of this, that has been
implemented, is an \proglang{XML}-based ``audit trail'' tracking the
history of operations applied to a data set. The conversion from DICOM
to ANALYZE/NIfTI is straightforward using the capabilities of
\pkg{oro.dicom}. The \proglang{S}4 classes have been developed to
provide a user-friendly interface to the ANALYZE/NIfTI data formats;
allowing easy data input, data output, image processing and
visualization.
}

\Keywords{export, imaging, import, medical, visualization}
\Plainkeywords{export, imaging, import, medical, visualization}

%% publication information
%% \Volume{50}
%% \Issue{9}
%% \Month{June}
%% \Year{2012}
%% \Submitdate{}
%% \Acceptdate{2012-06-04}

\Address{
    Brandon Whitcher\\
  Pfizer Worldwide R\&D\\
  Pfizer Worldwide Research \& Development 610 Main Street Cambridge, MA
  02139, United States\\
  E-mail: \email{bwhitcher@gmail.com}\\
  URL: \url{http://www.imperial.ac.uk/people/b.whitcher}\\~\\
      }

% Pandoc header

\usepackage{amsmath,rotating}

\begin{document}

\begin{CodeChunk}

\begin{CodeOutput}
oro.nifti 0.10.1
\end{CodeOutput}
\end{CodeChunk}

\hypertarget{introduction}{%
\section{Introduction}\label{introduction}}

Medical imaging is well established in both the clinical and research
areas with numerous equipment manufacturers supplying a wide variety of
modalities. The ANALYZE format was developed at the Mayo Clinic (in the
1990s) to store multidimensional biomedical images. It is fundamentally
different from the DICOM standard since it groups all images from a
single acquisition (typically three- or four-dimensional) into a pair of
binary files, one containing header information and one containing the
image information. The DICOM standard groups the header and image
information, typically a single two-dimensional image, into a single
file. Hence, a single acquisition will contain multiple DICOM files but
only a pair of ANALYZE files.

The NIfTI format was developed in the early 2000s by the DFWG (Data
Format Working Group) in an effort to improve upon the ANALYZE format.
The resulting NIfTI-1 format adheres to the basic header/image
combination from the ANALYZE format, but allows the pair of files to be
combined into a single file and re-defines the header fields. In
addition, NIfTI extensions allow one to store additional information
(e.g., key acquisition parameters, experimental design) inside a NIfTI
file.

\begin{table}[tbp]
  \begin{center}
    \begin{tabular}{p{0.475\textwidth}p{0.425\textwidth}}
      \hline
      \multicolumn{2}{c}{\pkg{oro.nifti}}\\
      \hline
      \code{afni}, \code{anlz}, \code{nifti} & Class constructors for AFNI, ANALYZE and NIfTI objects.\\
      \code{as(<obj>, "nifti")} & Coerce object into class \code{nifti}.\\
      \code{audit.trail}, \code{aux.file}, \code{descrip} & Extract or replace slots in specific header fields.\\
      \code{fmri2oro}, \code{oro2fmri} & Convert between \code{fmridata} (\pkg{fmri}) and \code{nifti} objects.\\
      \code{hotmetal}, \code{tim.colors} & Useful color tables for visualization.\\
      \code{image}, \code{orthographic}, \code{overlay} & Two-dimensional visualization methods.\\
      \code{is.afni}, \code{is.anlz}, \code{is.nifti} & Logical checks.\\
      \code{readAFNI}, \code{readANALYZE}, \code{readNIfTI} & Data input.\\
      \code{writeAFNI}, \code{writeANALYZE}, \code{writeNIfTI} & Data output.\\
      \hline
      \end{tabular}
  \end{center}
  \caption{List of functions available in \pkg{oro.nifti}.
    Functionality around the AFNI data format was recently added to
    the \pkg{oro.nifti} package.  Please visit
    \url{http://afni.nimh.nih.gov/afni} for more information about the
    AFNI data format.}
  \label{tab:functions}
\end{table}

The material presented here provides users with a method of interacting
with ANALYZE and NIfTI files in \proglang{R} \citep{R}. Real-world data
sets, that are publicly available, are used to illustrate the basic
functionality of \pkg{oro.nifti} \citep{whi-sch-tho:JSS}. It should be
noted that \pkg{oro.nifti} focuses on functions for data input/output
and visualization. \proglang{S}4 classes \texttt{nifti} and
\texttt{anlz} are provided for further statistical analysis in
\proglang{R} without losing contextual information from the original
ANALYZE or NIfTI files. Images in the metadata-rich DICOM format may be
converted to NIfTI semi-automatically using \pkg{oro.dicom} by utilizing
as much information from the DICOM files as possible. Basic
visualization functions, similar to those commonly used in the medical
imaging community, are provided for \texttt{nifti} and \texttt{anlz}
objects. Additionally, the \pkg{oro.nifti} package allows one to track
every operation on a \texttt{nifti} object in an \proglang{XML}-based
audit trail.

The \pkg{oro.nifti} package should appeal not only to \proglang{R}
package developers, but also to scientists and researchers who want to
interrogate medical imaging data using the statistical capabilities of
\proglang{R} without writing and validating their own basic data
input/output functionality. Table\textasciitilde{}\ref{tab:functions}
lists the key functions for \pkg{oro.nifti} and groups them according to
common functionality. An example of using statistical methodology in
\proglang{R} for the analysis of functional magnetic resonance imaging
(fMRI) data is given in section\textasciitilde{}\ref{fmri_example}.
Packages already available on CRAN that utilize \pkg{oro.nifti} include:
\pkg{cudaBayesreg} \citep{daSilva:JSS}, \pkg{dcemriS4}
\citep{whi-sch:JSS}, \pkg{dpmixsim} \citep{dpmixsim}, and
\pkg{RNiftyReg} \citep{RNiftyReg}.

\section[oro.nifti: NIfTI-1 data input/output in R]{\pkg{oro.nifti}: NIfTI-1 data input/output in \proglang{R}}

Although the industry standard for medical imaging data is DICOM,
another format has come to be heavily used in the image analysis
community. The ANALYZE format was originally developed in conjunction
with an image processing system (of the same name) at the Mayo
Foundation. A common version of the format, although not the most
recent, is called ANALYZE 7.5. A copy of the file ANALYZE75.pdf has been
included in \pkg{oro.nifti} (accessed via
\code{system.file("doc/ANALYZE75.pdf", package="oro.dicom")}) since it
does not appear to be available from \url{www.mayo.edu} any longer. An
ANALYZE 7.5 format image is comprised of two files, the ``.hdr'' and
``.img'' files, that contain information about the acquisition and the
acquisition itself, respectively. A more recent adaption of this format
is known as NIfTI-1 and is a product of the Data Format Working Group
(DFWG) from the Neuroimaging Informatics Technology Initiative (NIfTI;
\url{http://nifti.nimh.nih.gov}). The NIfTI-1 data format is almost
identical to the ANALYZE format, but offers a few improvements

\begin{itemize}
  \item merging of the header and image information
    into one file (.nii)
  \item re-organization of the 348-byte fixed header into more
    relevant categories 
  \item possibility of extending the header information.
\end{itemize}

There are several \proglang{R} packages that also offer input/output
functionality for the NIfTI and ANALYZE data formats in addition to
image analysis capabilities for specific MRI acquisition sequences;
e.g., \pkg{AnalyzeFMRI} \citep{AnalyzeFMRI}, \pkg{fmri}
\citep{pol-tab:fmri} and \pkg{tractor.base} \citep{tractor.base}. The
\pkg{Rniftilib} package provides access to NIfTI data via the
\proglang{nifticlib} library \citep{Rniftilib}.

\subsection{The NIfTI header}
\label{sec:nifti-header}

The NIfTI header inherits its structure (348 bytes in length) from the
ANALYZE data format. The last four bytes in the NIfTI header correspond
to the ``magic'' field and denote whether or not the header and image
are contained in a single file (\code{magic =
  "n+1\textbackslash{}0"}) or two separate files (\code{magic =
  "ni1\textbackslash{}0"}), the latter being identical to the structure
of the ANALYZE data format. The NIfTI data format added an additional
four bytes to allow for ``extensions'' to the header. By default these
four bytes are set to zero.

The first example of reading in, and displaying, medical imaging data in
NIfTI format \texttt{avg152T1\_LR\_nifti.nii.gz} was obtained from the
NIfTI website (\url{http://nifti.nimh.nih.gov/nifti-1/}). Successful
execution of the commands

\begin{CodeChunk}

\begin{CodeInput}
R> fname <- system.file(file.path("nifti", "mniLR.nii.gz"), package="oro.nifti")
R> (mniLR <- readNIfTI(fname))
\end{CodeInput}

\begin{CodeOutput}
NIfTI-1 format
  Type            : nifti
  Data Type       : 2 (UINT8)
  Bits per Pixel  : 8
  Slice Code      : 0 (Unknown)
  Intent Code     : 0 (None)
  Qform Code      : 0 (Unknown)
  Sform Code      : 4 (MNI_152)
  Dimension       : 91 x 109 x 91
  Pixel Dimension : 2 x 2 x 2
  Voxel Units     : mm
  Time Units      : sec
\end{CodeOutput}

\begin{CodeInput}
R> pixdim(mniLR)
\end{CodeInput}

\begin{CodeOutput}
[1] 0 2 2 2 1 1 1 1
\end{CodeOutput}

\begin{CodeInput}
R> descrip(mniLR)
\end{CodeInput}

\begin{CodeOutput}
[1] "FSL3.2beta"
\end{CodeOutput}

\begin{CodeInput}
R> aux.file(mniLR)
\end{CodeInput}

\begin{CodeOutput}
[1] "none                   "
\end{CodeOutput}
\end{CodeChunk}

produces an \proglang{S}4 \code{"nifti"} object (or
\code{"niftiAuditTrail"} if the audit trail option is set). Some
accessor functions are also provided; e.g., \code{aux.file} and
\code{descrip}. The former is used to access the original name of the
file (if it has been provided) and the latter is the name of a valid
NIfTI header field used to hold a ``description'' (up to 80 characters
in length).

\subsection{The NIfTI image}

Image information begins at the byte position determined by the
\code{voxoffset} slot. In a single NIfTI file (\code{magic =
  "n+1\textbackslash{}0"}), this is by default after the first 352
bytes. Header extensions extend the size of the header and come before
the image information leading to a consequent increase of
\code{voxoffset} for single NIfTI files. The split NIfTI (\code{magic
  = "ni1\textbackslash{}0"}) and ANALYZE formats contain pairs of files,
where the header and image information are separated, and do not have
this problem. In this case \code{voxoffset} is set to 0.

The \code{image} function has been overloaded so that it behaves
differently when dealing with medical image objects (\code{nifti} and
\code{anlz}). The command

\begin{CodeChunk}

\begin{CodeInput}
R> image(mniLR)
\end{CodeInput}
\end{CodeChunk}

produces a three-dimensional array of the MNI brain, with the default
NIfTI axes, and is displayed on a \(10{\times}10\) grid of images
(Figure \ref{fig:mniLR+mniRL}a). The \code{image} function for medical
image \proglang{S}4 objects is an attempt to balance minimal user input
with enough flexibility to customize the display when necessary. For
example, single slices may be viewed by using the option
\code{plot.type="single"} in conjunction with the option \code{z=} to
specify the slice.

\begin{figure}[tbp]
  \begin{center}
    \begin{tabular}{cc}
      \includegraphics*[width=0.45\textwidth]{mniLR.jpeg} &
      \includegraphics*[width=0.45\textwidth]{mniRL.jpeg}\\
      \textbf{(a)} & \textbf{(b)}
    \end{tabular}
  \end{center}
  \caption{\textbf{(a)} Axial slices of MNI volume \code{mniLR\_nifti}
    stored in the \emph{neurological} convention (right-is-right), but
    displayed in the \emph{radiological} convention (right-is-left).
    \textbf{(b)} Axial slices of MNI volume \code{mniRL\_nifti} stored
    and displayed in the \emph{radiological} convention.}
  \label{fig:mniLR+mniRL}
\end{figure}

The second example of reading in and displaying medical imaging data in
the NIfTI format \texttt{avg152T1\_RL\_nifti.nii.gz} was also obtained
from the NIfTI website (\url{http://nifti.nimh.nih.gov/nifti-1/}).
Successful execution of the commands

\begin{CodeChunk}

\begin{CodeInput}
R> fname <- system.file(file.path("nifti", "mniRL.nii.gz"), package="oro.nifti")
R> (mniRL <- readNIfTI(fname))
\end{CodeInput}

\begin{CodeOutput}
NIfTI-1 format
  Type            : nifti
  Data Type       : 2 (UINT8)
  Bits per Pixel  : 8
  Slice Code      : 0 (Unknown)
  Intent Code     : 0 (None)
  Qform Code      : 0 (Unknown)
  Sform Code      : 4 (MNI_152)
  Dimension       : 91 x 109 x 91
  Pixel Dimension : 2 x 2 x 2
  Voxel Units     : mm
  Time Units      : sec
\end{CodeOutput}
\end{CodeChunk}

\begin{CodeChunk}

\begin{CodeInput}
R> image(mniRL)
\end{CodeInput}
\end{CodeChunk}

produces a three-dimensional array of the MNI brain that is displayed in
a \(10{\times}10\) grid of images (Figure \ref{fig:mniLR+mniRL}b). The
two sets of data in Figure \ref{fig:mniLR+mniRL} are stored in two
different orientations, commonly referred to as the \emph{radiological}
and \emph{neurological} conventions. The neurological convention is
where ``right is right'' and one is essentially looking through the
subject. The radiological convention is where ``right is left'' and one
is looking at the subject.

\begin{figure}[tbp]
  \begin{center}
    \includegraphics*[width=0.65\textwidth]{mniRL_orthographic.jpeg}
    \end{center}
  \caption{Orthographic display of the MNI volume \code{mniRL\_nifti}.
    By default the mid-axial, mid-sagittal and mid-coronal planes are
    chosen.}
  \label{fig:mniRL-orthographic}
\end{figure}

An additional graphical display function has been added for \code{nifti}
and \code{anlz} objects that allows a so-called orthographic
visualization of the data.

\begin{CodeChunk}

\begin{CodeInput}
R> orthographic(mniRL)
\end{CodeInput}
\end{CodeChunk}

As seen in Figure \ref{fig:mniRL-orthographic} the mid-axial,
mid-sagittal and mid-coronal planes are displayed by default. The slices
used may be set using \code{xyz = c(I,J,K)}, where \((I,J,K)\) are
appropriate indices, and the crosshairs will provide a spatial reference
in each plane relative to the other two.

\subsection{A note on axes and orientation}

The NIfTI format contains an implicit generalized spatial transformation
from the data co-ordinate system \((i,j,k)\) into a real-space
``right-handed'' co-ordinate system. In this real-space system, the
\((x,y,z)\) axes are \emph{usually} set such that \(x\) increases from
left to right, \(y\) increases from posterior to anterior and \(z\)
increases from inferior to superior.

At this point in time the \pkg{oro.nifti} package cannot apply an
arbitrary transform to the imaging data into \((x,y,z)\) space -- such a
transform may require non-integral indices and interpolation steps. The
package does accommodate straightforward transformations of imaging
data; e.g., setting the \(i\)-axis to increase from right to left (the
neurological convention). Future versions of \pkg{oro.nifti} will
attempt to address more complicated spatial transformations and provide
functionality to display the \((x,y,z)\) axes on orthographic plots.

\subsection[NIfTI and ANALYZE data in S4]{NIfTI and ANALYZE data in \proglang{S}4}

A major improvement in the \pkg{oro.nifti} package is the fact that
standard medical imaging formats are stored in unique classes under the
\proglang{S}4 system \citep{chambers:2008}. Essentially, NIfTI and
ANALYZE data are stored as multi-dimensional arrays with extra slots
created that capture the format-specific header information; e.g., for a
\code{nifti} object

\begin{CodeChunk}

\begin{CodeInput}
R> slotNames(mniRL)
\end{CodeInput}

\begin{CodeOutput}
 [1] ".Data"          "sizeof_hdr"     "data_type"     
 [4] "db_name"        "extents"        "session_error" 
 [7] "regular"        "dim_info"       "dim_"          
[10] "intent_p1"      "intent_p2"      "intent_p3"     
[13] "intent_code"    "datatype"       "bitpix"        
[16] "slice_start"    "pixdim"         "vox_offset"    
[19] "scl_slope"      "scl_inter"      "slice_end"     
[22] "slice_code"     "xyzt_units"     "cal_max"       
[25] "cal_min"        "slice_duration" "toffset"       
[28] "glmax"          "glmin"          "descrip"       
[31] "aux_file"       "qform_code"     "sform_code"    
[34] "quatern_b"      "quatern_c"      "quatern_d"     
[37] "qoffset_x"      "qoffset_y"      "qoffset_z"     
[40] "srow_x"         "srow_y"         "srow_z"        
[43] "intent_name"    "magic"          "extender"      
[46] "reoriented"    
\end{CodeOutput}

\begin{CodeInput}
R> c(cal.min(mniRL), cal.max(mniRL))
\end{CodeInput}

\begin{CodeOutput}
[1]   0 255
\end{CodeOutput}

\begin{CodeInput}
R> range(mniRL)
\end{CodeInput}

\begin{CodeOutput}
[1]   0 255
\end{CodeOutput}

\begin{CodeInput}
R> mniRL@"datatype"
\end{CodeInput}

\begin{CodeOutput}
[1] 2
\end{CodeOutput}

\begin{CodeInput}
R> convert.datatype(mniRL@"datatype")
\end{CodeInput}

\begin{CodeOutput}
[1] "UINT8"
\end{CodeOutput}
\end{CodeChunk}

Note, an ANALYZE object has a slightly different set of slots. Slots
4--47 are taken verbatim from the definition of the NIfTI format and are
read directly from a file. The slot \code{.Data} is the multidimensional
array (since class \code{nifti} inherits from class \code{array}) and
the slots \code{trail}, \code{extensions} and \code{reoriented} are used
for internal bookkeeping. In the code above we have accessed the min/max
values of the imaging data using the \code{cal.min} and \code{cal.max}
accessor functions which matches a direct interrogation of the
\code{.Data} slot using the \code{range} function. Looking at the
\code{datatype} slot provides a numeric code that may be converted into
a value that indicates the type of byte structure used (in this case an
8-bit or 1-byte unsigned integer).

As introduced in Section\textasciitilde{}\ref{sec:nifti-header} there
are currently only two accessor functions to slots in the NIfTI header
(\code{aux.file} and \code{descrip}) -- all other slots are either
ignored or used inside of functions that operate on ANALYZE/NIfTI
objects. The NIfTI class also has the ability to read and write
extensions that conform to the NIfTI data format. Customized printing
and validity-checking functions are available to the user and every
attempt has been made to ensure that the information from the
multi-dimensional array is in agreement with the header values.

The constructor function \code{nifti} produces valid NIfTI objects,
including a consistent header, from an arbitrary array.

\begin{CodeChunk}

\begin{CodeInput}
R> n <- 100
R> (random.image <- nifti(array(runif(n*n), c(n,n,1))))
\end{CodeInput}

\begin{CodeOutput}
NIfTI-1 format
  Type            : nifti
  Data Type       : 2 (UINT8)
  Bits per Pixel  : 8
  Slice Code      : 0 (Unknown)
  Intent Code     : 0 (None)
  Qform Code      : 0 (Unknown)
  Sform Code      : 0 (Unknown)
  Dimension       : 100 x 100 x 1
  Pixel Dimension : 1 x 1 x 1
  Voxel Units     : Unknown
  Time Units      : Unknown
\end{CodeOutput}

\begin{CodeInput}
R> random.image@"dim_"
\end{CodeInput}

\begin{CodeOutput}
[1]   3 100 100   1   1   1   1   1
\end{CodeOutput}

\begin{CodeInput}
R> dim(random.image)
\end{CodeInput}

\begin{CodeOutput}
[1] 100 100   1
\end{CodeOutput}
\end{CodeChunk}

The function \code{writeNIfTI} outputs valid NIfTI class files, which
can be opened in other medical imaging software. Files can either be
stored as standard \code{.nii} files or compressed with gnuzip
(default).

\begin{CodeChunk}

\begin{CodeInput}
R> writeNIfTI(random.image, "random")
R> list.files(pattern="random")
\end{CodeInput}

\begin{CodeOutput}
[1] "random.nii.gz"
\end{CodeOutput}
\end{CodeChunk}

\subsection{The audit trail}

Following on from the \proglang{S}4 implementation of both the NIfTI and
ANALYZE data formats, the ability to extend the NIfTI data format header
is utilized in the \pkg{oro.nifti} package. Please use the command

\begin{CodeChunk}

\begin{CodeInput}
R> options(niftiAuditTrail=TRUE)
\end{CodeInput}
\end{CodeChunk}

to turn on the ``audit trail'' option in \pkg{oro.nifti} and then
execute the function \code{enableAuditTrail()}. With the option enabled
extensions are properly handled when reading and writing NIfTI data,
users are allowed to add extensions to newly-created NIfTI objects by
casting them as \code{niftiExtension} objects and adding
\code{niftiExtensionSection} objects to the \code{extensions} slot, and
all operations that are performed on a NIfTI object will generate what
we call an \emph{audit trail} that consists of an \proglang{XML}-based
log \citep{XML}.

Figure \ref{fig:mniLR-audit-trail} displays output from the accessor
function \code{audit.trail(mniLR)}, the \proglang{XML}-based audit trail
that is stored as a NIfTI header extension.

\begin{sidewaysfigure}
  \centering
  \begin{tabular}{p{22cm}}
```{=latex}
\begin{CodeChunk}

\begin{CodeInput}
R> audit.trail(mniLR)
\end{CodeInput}

\begin{CodeOutput}
NULL
\end{CodeOutput}
\end{CodeChunk}
```
  \end{tabular}
  \caption{\proglang{XML}-based audit trail obtained via
    \code{audit.trail(mniLR)}.  Note, this function will return
    \code{NULL} if the \pkg{XML} package is not available.}
  \label{fig:mniLR-audit-trail}
\end{sidewaysfigure}

Each log entry contains information not only about the function applied
to the NIfTI object, but also various system-level information; e.g.,
version of \proglang{R}, user name, date, time, etc. When writing
NIfTI-class objects to disk, the \proglang{XML}-based NIfTI extension is
converted into plain text and saved using \code{ecode=6} to denote plain
ASCII text. The user may control the tracking of data manipulation via
the audit trail using the global option \code{niftiAuditTrail}.

\subsection{Interactive visualization}

Basic visualization of \code{nifti} and \code{anlz} class images can be
achieved with any visualization for arrays in \proglang{R}. For example,
the \pkg{EBImage} package provides functions \code{display} and
\code{animate} for visualization \citep{EBImage}. Please note that
functions in \pkg{EBImage} expect grey-scale values in the range
\([0,1]\), hence the display of \code{nifti} data may be performed using

\begin{CodeChunk}

\begin{CodeInput}
R> mniLR.range <- range(mniLR)
R> EBImage::display((mniLR - min(mniLR)) / diff(mniLR.range))
\end{CodeInput}
\end{CodeChunk}

Interactive visualization of multi-dimensional arrays, stored in NIfTI
or ANALYZE format, is however best performed outside of \proglang{R} at
this point in time. Popular viewers, especially for neuroimaging data,
include

\begin{itemize}
\item FSLView (\url{http://www.fmrib.ox.ac.uk/fsl/fslview/}),
\item MRIcron (\url{http://cabiatl.com/mricron/}), 
\item ITKSnap (\url{http://www.itksnap.org}), and
\item VolView (\url{http://www.kitware.com/products/volview.html}).
\end{itemize}

The \pkg{mritc} package provides basic interactive visualization of
ANALYZE/NIfTI data using a \proglang{Tcl}/\proglang{Tk} interface
\citep{mritc}.

\subsection{An example using functional MRI data}
\label{fmri_example}

\begin{figure}[tbp]
  \begin{center}
    \begin{tabular}{c}
      \includegraphics*[width=0.6\textwidth]{ffd.jpeg}\\
      \textbf{(a)}\\
      \includegraphics*[width=0.6\textwidth]{ffd_orthographic.jpeg}\\
      \textbf{(b)}
    \end{tabular}
  \end{center}
  \caption{\textbf{(a)} Axial slices of the functional MRI data set
    \code{filtered\_func\_data} from the first acquisition.
    \textbf{(b)} Orthographic display of the first volume from the
    functional MRI data set \texttt{filtered\_func\_data}.}
  \label{fig:ffd+orthographic}
\end{figure}

This is an example of reading in, and displaying, a four-dimensional
medical imaging data set in NIfTI format \texttt{filtered\_func\_data}
obtained from the \pkg{FSL} evaluation and example data suite
(\url{http://www.fmrib.ox.ac.uk/fsl/fsl/feeds.html}). Successful
execution of the commands

\begin{CodeChunk}

\begin{CodeInput}
R> filtered.func.data <- 
R+   system.file(file.path("nifti", "filtered_func_data.nii.gz"), 
R+               package="oro.nifti")
R> (ffd <- readNIfTI(filtered.func.data))
\end{CodeInput}

\begin{CodeOutput}
NIfTI-1 format
  Type            : niftiExtension
  Data Type       : 4 (INT16)
  Bits per Pixel  : 16
  Slice Code      : 0 (Unknown)
  Intent Code     : 0 (None)
  Qform Code      : 0 (Unknown)
  Sform Code      : 0 (Unknown)
  Dimension       : 64 x 64 x 21 x 64
  Pixel Dimension : 1 x 1 x 1 x 1
  Voxel Units     : Unknown
  Time Units      : Unknown
\end{CodeOutput}
\end{CodeChunk}

\begin{CodeChunk}

\begin{CodeInput}
R> image(ffd, zlim=range(ffd)*0.95)
\end{CodeInput}
\end{CodeChunk}

produces a four-dimensional (4D) array of imaging data that may be
displayed in a \(5{\times}5\) grid of images (Figure
\ref{fig:ffd+orthographic}a). The first three dimensions are spatial
locations of the voxel (volume element) and the fourth dimension is time
for this functional MRI (fMRI) acquisition. As seen from the summary of
object, there are 21 axial slices of fairly coarse resolution
(\(4{\times}4{\times}6\;\text{mm}\)) and reasonable temporal resolution
(\(3\;\text{s}\)). Figure \ref{fig:ffd+orthographic}b depicts the
orthographic display of the \texttt{filtered\_func\_data} using the
axial plane containing the left-and-right thalamus to approximately
center the crosshair vertically.

\begin{CodeChunk}

\begin{CodeInput}
R> orthographic(ffd, xyz=c(34,29,10), zlim=range(ffd)*0.9)
\end{CodeInput}
\end{CodeChunk}

\subsubsection{Statistical analysis}

The \proglang{R} programming environment provides a wide variety of
statistical methodology for the quantitative analysis of medical imaging
data. For example, functional MRI (fMRI) data are typically analyzed by
applying a multiple linear regression model, commonly referred to in the
literature as a general linear model (GLM), that utilizes the stimulus
experiment to construct the design matrix. Estimation of the regression
coefficients in the GLM produces a statistical image; e.g.,
\(Z\)-statistics for a voxel-wise hypothesis test on activation in fMRI
experiments \citep{fri-etal:spms,fri-etal:revisited}.

\begin{figure}[tbp]
  \begin{center}
    \includegraphics*[width=\textwidth]{ffd_design.jpeg}
    \end{center}
  \caption{Visual (30 seconds on/off) and auditory (45 seconds on/off)
    stimuli, convolved with a parametric haemodynamic response
    function, used in the GLM-based fMRI analysis.}
  \label{fig:ffd-design}
\end{figure}

The 4D volume of imaging data in \texttt{filtered\_func\_data} was
acquired in an experiment with a repetition time
\(\text{TR}=3\;\text{s}\), using both visual and auditory stimuli. The
visual stimulus was applied using an on/off pattern for a duration of 60
seconds and the auditory stimulus was applied using an on/off pattern
for a duration of 90 seconds. A parametric haemodynamic response
function (HRF), with mean \(\mu=6\) and standard deviation \(\sigma=3\),
is utilized here which is similar to the default values in \pkg{FSL}
\citep{smi-etal:FSL}. We construct the experimental design and HRF in
seconds, perform the convolution and then downsample by a factor of
three in order to obtain columns of the design matrix that match the
acquisition of the MRI data.

\begin{CodeChunk}

\begin{CodeInput}
R> visual <- rep(c(-0.5,0.5), each=30, times=9)
R> auditory <- rep(c(-0.5,0.5), each=45, times=6)
R> hrf <- c(dgamma(1:15, 4, scale=1.5))
R> hrf0 <- c(hrf, rep(0, length(visual)-length(hrf)))
R> visual.hrf <- convolve(hrf0, visual)
R> hrf0 <- c(hrf, rep(0, length(auditory)-length(hrf)))
R> auditory.hrf <- convolve(hrf0, auditory)
R> index <- seq(3, 540, by=3)
R> visual.hrf <- visual.hrf[index]
R> auditory.hrf <- auditory.hrf[index]
\end{CodeInput}
\end{CodeChunk}

Figure \ref{fig:ffd-design} depicts the visual and auditory stimuli,
convolved with the HRF, in the order of acquisition. The design matrix
is then used in a voxel-wise GLM, where the \code{lsfit} function in
\proglang{R} estimates the parameters in the linear regression. At each
voxel \(t\)-statistics and their associated \(p\)-values are computed
for the hypothesis test of no effect for each individual stimulus, along
with an \(F\)-statistic for the hypothesis test of no effect of any
stimuli using the \code{ls.print} function.

\begin{CodeChunk}

\begin{CodeInput}
R> ##reduced length due to R package storage limitations
R> visual.hrf<-visual.hrf[1:64]
R> auditory.hrf<-auditory.hrf[1:64]
R> ## background threshold: 10% max intensity
R> voxel.lsfit <- function(x, thresh) { # general linear model
R+   ## check against background threshold
R+   if (max(x) < thresh) {
R+     return(rep(NA, 5))
R+   }
R+   ## glm
R+   output <- lsfit(cbind(visual.hrf, auditory.hrf), x)
R+   ## extract t-statistic, p-values
R+   output.t <- ls.print(output, print.it=FALSE)$coef.table[[1]][2:3,3:4]
R+   output.f <- ls.print(output, print.it=FALSE)$summary[3]
R+   c(output.t, as.numeric(output.f))
R+ }
R> ## apply local glm to each voxel
R> ffd.glm <- apply(ffd, 1:3, voxel.lsfit, thresh=0.1 * max(ffd))
\end{CodeInput}
\end{CodeChunk}

\begin{figure}[tbp]
  \begin{center}
    \begin{tabular}{c}
      \includegraphics*[width=0.6\textwidth]{ffd_zstat1.jpeg}\\
      \textbf{(a)}\\
      \includegraphics*[width=0.6\textwidth]{ffd_zstat2.jpeg}\\
      \textbf{(b)}
    \end{tabular}
  \end{center}
  \caption{\textbf{(a)} Axial slices of the functional MRI data with
    the statistical image from the visual stimulus overlayed.
    \textbf{(b)} Axial slices of the functional MRI data with the
    statistical image from the auditory stimulus overlayed.  Both sets
    of test statistics were thresholded at $Z\geq{5}$ for all voxel.}
  \label{fig:zstat1+zstat2}
\end{figure}

Given the multidimensional array of output from the GLM fitting
procedure, the \(t\)-statistics are separated and converted into
\(Z\)-statistics to follow the convention used in \pkg{FSL}. For the
purposes of this example we have not applied any multiple comparisons
correction procedure and, instead, use a relatively large threshold of
\(Z>5\) for visualization.

\begin{CodeChunk}

\begin{CodeInput}
R> dof <- ntim(ffd) - 1
R> Z.visual <- nifti(qnorm(pt(ffd.glm[1,,,], dof, log.p=TRUE), log.p=TRUE),
R+                   datatype=16)
R> Z.auditory <- nifti(qnorm(pt(ffd.glm[2,,,], dof, log.p=TRUE), log.p=TRUE),
R+                     datatype=16)
\end{CodeInput}
\end{CodeChunk}

\begin{CodeChunk}

\begin{CodeInput}
R> yrange <- c(5, max(Z.visual, na.rm=TRUE))
R> overlay(ffd, ifelse(Z.visual > 5, Z.visual, NA), 
R>         zlim.x=range(ffd)*0.95, zlim.y=yrange)
\end{CodeInput}
\end{CodeChunk}

\begin{CodeChunk}

\begin{CodeInput}
R> yrange <- c(5, max(Z.auditory, na.rm=TRUE))
R> overlay(ffd, ifelse(Z.auditory > 5, Z.auditory, NA), 
R>         zlim.x=range(ffd)*0.95, zlim.y=yrange)
\end{CodeInput}
\end{CodeChunk}

Statistical images in neuroimaging are commonly displayed as an overlay
on top of a reference image (one of the dynamic acquisitions) in order
to provide anatomical context. The \code{overlay} command in
\pkg{oro.nifti} allows one to display the statistical image of
voxel-wise activations overlayed on one of the original EPI (echo planar
imaging) volumes acquired in the fMRI experiment. The 3D array of
\(Z\)-statistics for the visual and auditory tasks are overlayed on the
original data for ``anatomical'' reference in Figure
\ref{fig:zstat1+zstat2}. The \(Z\)-statistics that exceed the threshold
appear to match know neuroanatomy, where the visual cortex in the
occipital lobe shows activation under the visual stimulus (Figure
\ref{fig:zstat1+zstat2}a) and the primary auditory cortex in the
temporal lobe shows activation under the auditory stimulus (Figure
\ref{fig:zstat1+zstat2}b).

\section{Conclusion}

Medical image analysis depends on the efficient manipulation and
conversion of DICOM data. The \pkg{oro.nifti} package has been developed
to provide the user with a set of functions that mask as many of the
background details as possible while still providing flexible and robust
performance.

The future of medical image analysis in \proglang{R} will benefit from a
unified view of the imaging data standards: DICOM, NIfTI, ANALYZE, AFNI,
MINC, etc. The existence of a single package for handling imaging data
formats would facilitate interoperability between the ever increasing
number of \proglang{R} packages devoted to medical image analysis. We do
not assume that the data structures in \pkg{oro.nifti} are best-suited
for this purpose and we welcome an open discussion around how best to
provide this standardization to the end user.

\section*{Acknowledgments}

The authors would like to thank the National Biomedical Imaging Archive
(NBIA), the National Cancer Institute (NCI), the National Institute of
Health (NIH) and all institutions that have contributed medical imaging
data to the public domain. The authors would also like to thank K.
Tabelow for providing functionality around the AFNI data format. VS is
supported by the German Research Council (DFG SCHM 2747/1-1).

\bibliography{nifti.bib}


\end{document}

